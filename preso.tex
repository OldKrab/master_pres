%% Простая презентация с примером включения программного кода и
%% пошаговых спецэффектов
\documentclass[aspectratio=169]{beamer}
\usetheme{SPbAU}
%\useoutertheme{infolines}
\usepackage{fontspec}
\usepackage{xunicode}
\usepackage{xltxtra}
\usepackage{xecyr}
\usepackage{hyperref}
\setmainfont[Mapping=tex-text]{DejaVu Serif}
\setsansfont[Mapping=tex-text]{DejaVu Sans}
\setmonofont[Mapping=tex-text]{DejaVu Sans Mono}
\usepackage{polyglossia}
\setdefaultlanguage{russian}
\usepackage{graphicx}
\usepackage[outputdir=out]{minted}
\usepackage{multicol}
\usepackage{array}
\usepackage{svg}
\usepackage[table]{xcolor}
\usepackage{multirow}
\usepackage{url}
\usepackage{listings}
\setminted[c++]{
  linenos=true,
  fontsize=\footnotesize
}
\lstdefinestyle{mycode}{
  belowcaptionskip=1\baselineskip,
  breaklines=true,
  xleftmargin=\parindent,
  showstringspaces=false,
  basicstyle=\footnotesize\ttfamily,
  keywordstyle=\bfseries,
  commentstyle=\itshape\color{gray!40!black},
  stringstyle=\color{red},
  numbers=left,
  numbersep=5pt,
  numberstyle=\tiny\color{gray},
}
\lstset{escapechar=@,style=mycode}

\definecolor{myred}{RGB}{255, 0, 0}
\definecolor{myblue}{RGB}{0, 0, 255}
\definecolor{myviolet}{RGB}{142, 0, 255}


\setbeamertemplate{navigation symbols}{} % Убрать навигацию

\begin{document}
\title[Комбинаторы запросов к графам]
{
  Разработка библиотеки комбинаторов запросов к графам
}
\author[Шушаков Д.С.]
{
Шушаков Даниил Сергеевич\\
{\footnotesize\textcolor{gray}{Научный руководитель: Григорьев Семен Вячеславович}}\\
}
\institute{Университет ИТМО}


\date{\today}
\frame{\titlepage\thispagestyle{empty}}

\setlength{\parskip}{0.25cm}


%=============================================

\begin{frame}{Графовое представление данных}
  \begin{itemize}
    \item Данные могут представляться в виде графа
    \item Хранятся в графовых базах данных
    \item Вершины и ребра могут иметь свойства вида <<ключ: значение>>
  \end{itemize}

  \center{\includegraphics[scale=0.4]{images/gr_example2.png}}
\end{frame}


%=============================================

\begin{frame}[fragile]{Языки запросов}
  \begin{itemize}
    \item Для доступа к данным используются языки запросов, например, Cypher для Neo4j\footnote{https://neo4j.com/docs/getting-started/get-started-with-neo4j/graph-database}
    \item В пользовательском коде они могут использоваться как строки:
          \begin{minted}[frame=single,framesep=10pt, fontsize=\small]{java}
tx.execute("MATCH (n {name: 'my node'}) RETURN n, n.name")
\end{minted}
    \item Недостатки:
          \begin{itemize}
            \item Отсутствие проверки корректности запроса на этапе компиляции
            \item Сложность составления и переиспользования подзапросов
          \end{itemize}
    \item Удобно иметь бесшовную интеграцию запросов в язык программирования
  \end{itemize}

  % Result result = tx.execute( "MATCH (n {name: 'my node'}) RETURN n, n.name" ) )
\end{frame}


%=============================================

\begin{frame}[fragile]{Применение парсер-комбинаторов к графам}

  \begin{columns}[c]
    \begin{column}{0.45\textwidth}
      \begin{itemize}
        \item Запрос к графу --- нахождение путей, удовлетворяющих ограничениям
        \item Ограничения на пути можно описать с помощью грамматики
        \item Грамматику в ЯП общего назначения удобно описать через парсер-комбинаторы
      \end{itemize}
    \end{column}
    \begin{column}{0.58\textwidth}
      \center{\includegraphics[scale=0.43]{images/gr_query.png}}
      \begin{minted}[frame=single,fontsize=\small, escapeinside=||]{text}
val john = v { it.name == "John" }
val pet = outE { it.label == "PET" }
val s = |\textcolor{myred}{john}| seq |\textcolor{myblue}{pet}| seq |\textcolor{myviolet}{outV()}|
\end{minted}

    \end{column}
  \end{columns}

\end{frame}

%=============================================

\begin{frame}[fragile]{Дополнительные требования}
  \begin{itemize}
    \item Описывать КС-ограничения: существуют применения в области биоинформатики\footnote[1]{Sevon et al (2016). Subgraph Queries by Context-free Grammars}, статическом анализе\footnote[2]{Reps (1997). Program analysis via graph reachability} и т.д.
    \item Разбирать циклы: результатов может быть бесконечно много
  \end{itemize}
  \begin{center}
    \center{\includegraphics[scale=0.35]{images/gr_friend_loop.png}}
  \end{center}
  %   \begin{minted}[frame=single,fontsize=\small]{kotlin}
  % val s = friend or (s seq friend)
  % \end{minted}
\end{frame}


%=============================================

\begin{frame}{Аналоги}
  \begin{table}[h]
    \renewcommand{\arraystretch}{1.3}
    \begin{tabular}{|b{0.25\linewidth}|b{0.33\linewidth}|b{0.33\linewidth}|}
      \hline
      \textbf{Свойства} & \textbf{Trails\footnote[1]{Kr\"{o}ni et al (2013). Parsing graphs: applying parser combinators to graph traversals}} & \textbf{Meerkat\footnote[2]{\href{https://dl.acm.org/doi/10.1145/2847538.2847539}
          {Izmaylova et al (2016). Practical, General Parser Combinators}}} \\
      \hline
      Сложность      & Экспоненциальная                  & Полиномиальная \\
      \hline
      Левая рекурсия & Нет                               & Да \\
      \hline
      Циклы в графе  & Ограничено                        & Да \\
      \hline
      Типизация      & Контролирует корректность парсера & Возможны некорректные парсеры \\
      \hline
      ЯП             & Scala                             & Scala \\
      \hline
    \end{tabular}
  \end{table}
\end{frame}



%=============================================

\begin{frame}{Подходы к реализации}
  \begin{enumerate}
    \item Meerkat основан на подходах из статьи Practical, General Parser Combinators\footnote[1]{\href{https://dl.acm.org/doi/10.1145/2847538.2847539}
            {Izmaylova et al (2016). Practical, General Parser Combinators}}
          \begin{enumerate}
            \item Мемоизация результатов парсера --- левой рекурсия в грамматиках
            \item Генерация сжатого представления деревьев разбора (SPPF) --- циклы в графах и полиномиальное время
          \end{enumerate}
    \item Trails имеет удобные комбинаторы, контролирующие корректность парсеров
  \end{enumerate}
  % \begin{enumerate}
  %   \item Meerkat основан на подходах из статьи\footnote[1]{Izmaylova et al (2016). Practical, General Parser Combinators}
  %   \item Для поддержки левой рекурсии в грамматиках добавляется мемоизация результатов парсера
  %   \item В процессе разбора генерируется сжатое представление деревьев разбора (SPPF)
  %   \item Оба подхода позволяют получить:
  %         \begin{enumerate}
  %           \item Полиномиальное время работы
  %           \item Поддержку циклов в графе
  %         \end{enumerate}
  % \end{enumerate}
\end{frame}



%=============================================

\begin{frame}{Цель и задачи}
  \textbf{Цель:} разработать библиотеку комбинаторов запросов к графам на основе алгоритмов, предложенных в работе Practical, General Parser Combinators, позволяющую описывать корректные парсеры.

  \textbf{Задачи:}
  \begin{enumerate}
    \item Разработать базовую структуру парсера и комбинаторов с корректной типизацией
    \item Поддержать левую рекурсию в грамматике
    \item Поддержать разбор циклов в графе
    \item Сравнить скорость работы с другими решениями
  \end{enumerate}
\end{frame}


%=============================================

\begin{frame}[fragile]{Реализация: базовая структура парсера}
  Решение разработано на языке Kotlin.
  \begin{minted}[frame=single,framesep=10pt, fontsize=\small]{kotlin}
type Parser = (In) -> ParserResult<Out, Res>
  \end{minted}
  \begin{itemize}
    \item Парсер принимает входящее состояние, возвращает множество выходящий состояний и результатов
    \item Состояние: вершина, ребро, стартовое
    \item Базовые парсеры: \texttt{v, edge, inE, outE, inV, outV}
    \item Комбинаторы: \texttt{seq, or, many и т.д.}
    \item Комбинаторы запрещают объединять парсеры с конфликтующими типами состояний
  \end{itemize}

  %   \begin{minted}[frame=single, fontsize=\small]{kotlin}
  % val person = v()
  % val loves = outE { it.label == "loves" }
  % val friend = outE { it.label == "friend" }
  % val p = person seq ((friend or loves) seq outV()).some
  %   \end{minted}
\end{frame}



%=============================================

\begin{frame}[fragile]{Левая рекурсия: мемоизация результатов}
  \begin{itemize}
    \item Разбор может бесконечно выполняться: левая рекурсия

    \item Хотим, чтобы вычисление парсера исполнялось один раз

    \item Парсер в качестве результата теперь возвращает вычисление:
          \begin{minted}[frame=single,framesep=4pt, fontsize=\small, escapeinside=||]{text}
type Parser = (In) -> (Continuation|\footnote[1]{Можно воспринимать как callback}|<Out, Res>) -> Unit
\end{minted}
    \item Парсер мемоизирует вычисления для каждого входящего состояния

    \item Вычисление запоминает все результаты и продолжения, при появлении нового результата вызывает эти продолжения

    \item Всем продолжениям будет переданы все результаты вычисления

  \end{itemize}
\end{frame}


%=============================================

\begin{frame}{Циклы в графе: генерация SPPF}
  \begin{columns}[c]
    \begin{column}{0.55\textwidth}
      \begin{itemize}
        \item Разбор графа с циклами потенциально может генерировать бесконечно много результатов
        \item Нужно представить результат такого разбора в виде конечной структуры --- SPPF
        \item Результаты из SPPF извлекаются обходом
      \end{itemize}

      \vspace{1cm}

      \begin{minipage}[t]{0.49\textwidth}
        \center{\Large $S \to \varepsilon\ |\ S\ a\ x$}

        \centering Грамматика
      \end{minipage}
      \begin{minipage}[t]{0.49\textwidth}
        \center{\includesvg[inkscapelatex=false,width=0.55\textwidth]{images/gr_loop.svg}}

        \centering Граф-петля
      \end{minipage}%

    \end{column}
    \begin{column}{0.55\textwidth}
      \center{\includesvg[inkscapelatex=false,width=0.75\textwidth]{images/loop.svg}}

      \centering Полученный SPPF
    \end{column}
  \end{columns}
\end{frame}


%=============================================

\begin{frame}[fragile]{Набор данных для исследования времени работы}
  \begin{itemize}
    \item Для исследования добавлена поддержка базы данных Neo4j
    \item Взяты три набора данных: онтологии\footnote[1]{Zhang et al (2016). Context-free path queries on RDF graphs}, SFPQ\_Data\footnote[2]{https://formallanguageconstrainedpathquerying.github.io/CFPQ\_Data/graphs}, Yago\footnote[3]{https://gitlab.inria.fr/tyrex-public/rlqdag}
    \item Для первых двух использовался одинаковый КС-запрос
          \begin{minted}[frame=single,fontsize=\small, escapeinside=||]{text}
|$Q_1 \to subclassof^{-1}\ Q_1?\ subclassof$|
|$Q_1 \to type^{-1}\ Q_1?\ type$|
\end{minted}
    \item Для Yago использовался регулярный запрос
          \begin{minted}[frame=single,fontsize=\footnotesize]{text}
MATCH 
(x)-[:P1]->()-[:P2]->()-[:P3*1..]->()-[:P4*1..]->(:Entity{id:'42'}) 
RETURN DISTINCT x
\end{minted}
          % MATCH (x)-[:P74636308]->()-[:P59561600]->()-[:P13305537*1..]->
          % ()-[:P92580544*1..]->(:Entity{id:'40324616'}) RETURN DISTINCT x
  \end{itemize}

\end{frame}


%=============================================

\begin{frame}{Результаты исследования времени работы}
  \vspace*{-0.9\baselineskip}
  \begin{table}[h]
    \renewcommand{\arraystretch}{1.3}
    \begin{tabular}{|b{0.02\linewidth}|b{0.21\linewidth}|b{0.15\linewidth}|b{0.15\linewidth}|>{\columncolor{yellow!20}}b{0.15\linewidth}|b{0.14\linewidth}|}
      \hline
      \multicolumn{2}{|c|}{\textbf{Граф}}        & \textbf{Кол-во вершин}    & \textbf{Кол-во рёбер} & \textbf{Мое реш. (мс)} & \textbf{Meerkat (мс)} \\
      \hline
      \multirow{4}{*}{\rotatebox{90}{онтологии}} & atom-primitive            & 291        & 685        & \textbf{19.5} & 22.0 \\
                                                 & b.-m.-p.\footnote[1]{biomedical-mesure-primitive} & 341        & 711        & \textbf{23.5} & 42.5 \\
                                                 & generations               & 129        & 351        & \textbf{1.3}  & 1.4 \\
                                                 & pizza                     & 671        & 2,604      & \textbf{61.4} & 123.9 \\
      \hline

      \multirow{4}{*}{\rotatebox{90}{SFPQ\_Data}} & enzyme                    & 48,815     & 86,543     & 121           & \textbf{116} \\
                                                 & eclass                    & 239,111    & 360,248    & 1220          & \textbf{1095} \\
                                                 & go                        & 582,929    & 1,437,437  & 6231          & \textbf{5367} \\
                                                 & go\_hierarchy             & 45,007     & 490,109    & 12015         & \textbf{8337} \\
      \hline
      \multicolumn{1}{|l}{}                      & \multicolumn{1}{l|}{yago} & 42,832,856 & 62,643,951 & \textbf{5558} & OOM \\
      \hline
    \end{tabular}
  \end{table}
\end{frame}



% Meerkat, Gparsers
% atom-primitive.owl,38,50
% biomedical-mesure-primitive.owl,48,50
% eclass.csv,310,398
% enzyme.csv,82,102
% foaf.rdf,24,32
% generations.owl,22,30
% geospecies.csv,1466,722
% go.csv,983,920
% go_hierarchy.csv,266,96
% people_pets.rdf,31,36
% pizza.owl,90,90
% skos.rdf,22,30
% travel.owl,24,30
% univ-bench.owl,24,30
% wine.rdf,100,90

%=============================================

\begin{frame}{Результаты}
  \begin{itemize}
    \item Разработаны базовая структура парсера и комбинаторов, проверяющая корректность во время компиляции
    \item Поддержаны леворекурсивные грамматики
    \item Поддержан разбор циклов в графе
    \item Реализована базовая поддержка Neo4j графов
    \item Проведено исследование времени работы в сравнении с Meerkat:
          \begin{itemize}
            \item На первом датасете мое решение быстрее в среднем на 28\%
            \item На втором датасете Meerkat быстрее в среднем на 15\%
            \item На третьем датасете только мое решение успешно выполнилось
          \end{itemize}
  \end{itemize}
\end{frame}


% --------- ДОП СЛАЙДЫ -------------
\appendix


\begin{frame}{Аналоги: Meerkat}
  \begin{itemize}
    \item Meerkat --- изначально библиотека с комбинаторами для разбора текста, реализующий алгоритм из статьи \footnote[1]{\href{https://dl.acm.org/doi/10.1145/2847538.2847539}
            {Izmaylova et al (2016). Practical, General Parser Combinators}}
    \item В последствии расширена поддержкой запросов к графам\footnote[2]{Verbitskaia et al (2018). Parser combinators for context-free path querying}
    \item Т.к. изначально разрабатывалась для разбора текста, в ней парсеры принимали позицию в тексте
    \item В расширении позиция в тексте семантически поменялась на номер вершины в графе
  \end{itemize}
\end{frame}

\begin{frame}[fragile]{Сигнатуры комбинаторов}

  \begin{minted}[frame=single,fontsize=\footnotesize]{kotlin}
fun BaseParser<I, O1, R1>.seq(p2: BaseParser<O1, O2, R2>)
    : BaseParser<I, O2, Pair<R1, R2>>

fun BaseParser<I, O, R>.or(p2: BaseParser<I, O, R>)
    : BaseParser<I, O, R>

fun rule(first: BaseParser<I, O, R>, vararg rest: BaseParser<I, O, R>)

fun BaseParser<I, O, A>.using(f: (A) -> B): BaseParser<I, O, B>

fun BaseParser<I, O, R>.that(constraint: BaseParser<O, O2, R2>)
    : BaseParser<I, O, R> 
    
val BaseParser<S, S, R>.many: BaseParser<S, S, List<R>>
val BaseParser<S, S, R>.some: BaseParser<S, S, List<R>>
\end{minted}
\end{frame}


\begin{frame}[fragile]{Комбинаторы that и using}
  Использование комбинаторов \texttt{that, using}:
  \begin{minted}[frame=single, fontsize=\small]{kotlin}
val mary = outV { it.value == "Mary" }
val loves = outE { it.label == "loves" }
val friend = outE { it.label == "friend" }
val maryLover = v().that(loves seq mary)
val p = maryLover seqr (friend seq outV()).some
val s = p using {edges -> Pair(edges.size, edges.last().second)}
\end{minted}
\end{frame}


\begin{frame}[fragile]{Запросы для исследования времени (КС)}
  \begin{minted}[frame=single, fontsize=\footnotesize]{kotlin}
val subclassof1 = throughInE { it.label == "subClassOf" }
val subclassof = throughOutE { it.label == "subClassOf" }
val type1 = throughInE { it.label == "type" }
val type = throughOutE { it.label == "type" }
val Q1 = LazyParser<Neo4jVertexState, Neo4jVertexState, Any>()
Q1.p = (subclassof1 seq (Q1 or eps()) seq subclassof) or
        (type1 seq (Q1 or eps()) seq type)
\end{minted}
  \begin{minted}[frame=single, fontsize=\footnotesize]{scala}
val subclassof1: Symbol[L, N, _] = inE("subClassOf")
val subclassof: Symbol[L, N, _] = outE("subClassOf")
val type1: Symbol[L, N, _] = inE("type")
val _type: Symbol[L, N, _] = outE("type")
val grammar: Symbol[L, N, _] = 
  syn((subclassof1 ~ syn(grammar | ε) ~ subclassof) |
    (type1 ~ syn(grammar | ε) ~ _type))
\end{minted}


\end{frame}


\begin{frame}[fragile]{Запросы для исследования времени (yago)}


  \begin{minted}[frame=single,fontsize=\footnotesize]{kotlin}
(v { it.properties["id"] == "40324616" } seq
(inE { it.label == "P92580544" } seq inV()).some seq
(inE { it.label == "P13305537" } seq inV()).some seq
inE { it.label == "P59561600" } seq inV() seq
inE { it.label == "P74636308" } seq inV())
\end{minted}

  \begin{minted}[frame=single, fontsize=\footnotesize]{scala}
syn(V((e: Entity) => e.getProperty("id") == "40324616") ~
inE((e: Entity) => e.label() == "P92580544").+ ~
inE((e: Entity) => e.label() == "P13305537").+ ~
inE((e: Entity) => e.label() == "P59561600") ~
inE((e: Entity) => e.label() == "P74636308"))
  \end{minted}

\end{frame}

\begin{frame}{Результаты исследования потребления памяти}
  \vspace*{-0.9\baselineskip}
  \begin{table}[h]
    \renewcommand{\arraystretch}{1.3}
    \begin{tabular}{|b{0.02\linewidth}|b{0.21\linewidth}|b{0.15\linewidth}|b{0.15\linewidth}|>{\columncolor{yellow!20}}b{0.15\linewidth}|b{0.14\linewidth}|}
      \hline
      \multicolumn{2}{|c|}{\textbf{Граф}}        & \textbf{Кол-во вершин}    & \textbf{Кол-во рёбер} & \textbf{Мое реш. (мб)} & \textbf{Meerkat (мб)} \\
      \hline
      \multirow{4}{*}{\rotatebox{90}{онтологии}} & atom-primitive            & 291        & 685        & 50 & \textbf{38} \\
                                                 & b.-m.-p.\footnote[1]{biomedical-mesure-primitive} & 341        & 711        & 50 & \textbf{48} \\
                                                 & generations               & 129        & 351        & 30  & \textbf{22} \\
                                                 & pizza                     & 671        & 2,604      & 90 & 90 \\
      \hline

      \multirow{4}{*}{\rotatebox{90}{SFPQ\_Data}} & enzyme                    & 48,815     & 86,543     & 102           & \textbf{82} \\
                                                 & eclass                    & 239,111    & 360,248    & 398          & \textbf{310} \\
                                                 & go                        & 582,929    & 1,437,437  & \textbf{920}          & 983 \\
                                                 & go\_hierarchy             & 45,007     & 490,109    & \textbf{96}         & 266 \\
      \hline
    \end{tabular}
  \end{table}
\end{frame}

\begin{frame}{Время работы от плотности меток}
  \vspace*{-0.9\baselineskip}
  \begin{table}[h]
    \renewcommand{\arraystretch}{1.3}
    \begin{tabular}{|b{0.02\linewidth}|b{0.21\linewidth}|b{0.15\linewidth}|b{0.17\linewidth}|b{0.15\linewidth}|b{0.15\linewidth}|}
      \hline
      \multicolumn{2}{|c|}{\textbf{Граф}}        & \textbf{Кол-во меток type} & \textbf{Кол-во меток subclassof} & \textbf{Кол-во ребер} & \textbf{Мое реш. (мс)} \\
      \hline
      \multirow{4}{*}{\rotatebox{90}{онтологии}} & atom-primitive             & 138     & 122     &  685 & 19.5 \\
                                                 & b.-m.-p.\footnote[1]{biomedical-mesure-primitive} & 130     & 122     & 711 & 23.5 \\
                                                 & generations                & 78      & 0       & 351 & 1.3 \\
                                                 & pizza                      & 365     & 259     & 2,604 & 61.4 \\
      \hline

      \multirow{4}{*}{\rotatebox{90}{SFPQ\_Data}} & enzyme                     & 14,989  & 8,163   & 86,543 & 121 \\
                                                 & eclass                     & 72,517  & 90,962  & 360,248 & 1220 \\
                                                 & go                         & 226,481 & 94,514  & 1,437,437 & 6231 \\
                                                 & go\_hierarchy              & 0       & 490,109 & 490,109 & 12015 \\
      \hline
    \end{tabular}

  \end{table}
\end{frame}



\begin{frame}{Хранение SPPF узлов}
  \begin{itemize}
    \item Если одинаковые узлы создаются в разных парсерах, то это должны быть одинаковые объекты
    \item Поэтому используется единое хранилище узлов SPPF
    \item Хранилище позволяет либо создать новый узел, либо вернуть существующий
    \item Хранилище общее для всех парсеров во время одного разбора
    \item Можно попробовать хранить узлы локально в каждом парсере, но нужно убедится в корректности такого подхода
  \end{itemize}
\end{frame}

\begin{frame}{Исходный код}
 Исходный код доступен в репозитории GitHub: \url{https://github.com/OldKrab/kotlin_graph_parsers}
\end{frame}

\end{document}

